\documentclass[12pt]{article}
\usepackage{fancyhdr}
\usepackage[letterpaper, margin=1in]{geometry}
%\usepackage{indentfirst}
\usepackage{graphicx}
\usepackage{amsmath}
\usepackage{amssymb}
\usepackage{siunitx}
\sisetup{detect-weight=true, detect-family=true} % makes siunitx follow font formatting like bold, italic, etc.
\usepackage{cancel}
\usepackage{isotope}
\usepackage{listings}
\usepackage[dvipsnames,table]{xcolor}
\usepackage{xspace}
\usepackage{booktabs} % makes tables pretty
\usepackage{longtable} % for long tables
\usepackage{multirow} % makes tables pretty
\usepackage{multicol} % makes tables pretty
\usepackage{setspace}
\usepackage{subcaption}
\usepackage{hyperref}
\usepackage{cleveref}
\newcommand{\creflastconjunction}{, and\nobreakspace} % adds oxford comma to cleveref
\usepackage[utf8]{inputenc}
\usepackage{textcomp}
\usepackage{titlesec}
\usepackage{svg}
\usepackage{pdflscape} % makes pages landscape
\usepackage{mathtools}
\usepackage{enumitem}
\usepackage[T1]{fontenc}


% ----------------- Commands ---------------------%
\newcommand{\mathcolorbox}[2]{\colorbox{#1}{$\displaystyle #2$}}
\definecolor{codegreen}{rgb}{0,0.6,0}
\definecolor{codegray}{rgb}{0.5,0.5,0.5}
\definecolor{codepurple}{rgb}{0.58,0,0.82}
\definecolor{backcolour}{rgb}{0.95,0.95,0.92}

% listings stuff for python
% Default fixed font does not support bold face
\DeclareFixedFont{\ttb}{T1}{txtt}{bx}{n}{12} % for bold
\DeclareFixedFont{\ttm}{T1}{txtt}{m}{n}{12}  % for normal

% Custom colors
%\usepackage{color}
\definecolor{deepblue}{rgb}{0,0,0.5}
\definecolor{deepred}{rgb}{0.6,0,0}
\definecolor{deepgreen}{rgb}{0,0.5,0}

%\usepackage{listings}

% Python style for highlighting
\newcommand\pythonstyle{\lstset{
language=Python,
basicstyle=\ttm,
morekeywords={self},              % Add keywords here
keywordstyle=\ttb\color{deepblue},
emph={MyClass,__init__},          % Custom highlighting
emphstyle=\ttb\color{deepred},    % Custom highlighting style
stringstyle=\color{deepgreen},
commentstyle=\color{codegray},
backgroundcolor = \color{backcolour},
breaklines=true,
numbers=left,
numberstyle=\small\color{codegray},
frame=tb,                         % Any extra options here
showstringspaces=false
}}

% Python for external files
\newcommand\pythonexternal[2][]{{
\pythonstyle
\lstinputlisting[#1]{#2}}}




% si units stuff
\DeclareSIUnit\year{yr}
\DeclareSIUnit\hour{hr}
\DeclareSIUnit\mole{mol}

% fancy header stuff
\usepackage{fancyhdr}
\pagestyle{fancy}

\setlength{\headheight}{28pt}
\lhead{NSE 565 \\ Winter 2022}
\chead{Homework 2\\}
\rhead{Austin Warren\\Due March 11, 2022}

% bib if needed
\bibliographystyle{ieeetr}

\begin{document}

%%%%%%%%%%%%%%%%%%%%%%%%%%%%%%%%%%
\section{Methods}
We will use the Upwind scheme for the spatial discretization. The conditions for the Upwind scheme are listed below:
\begin{equation}
    \phi_{i+1} = \begin{cases}
                    \phi_{I} & \overline{V} > 0\\
                    \phi_{I+1} & \overline{V} < 0
                 \end{cases}\:,
    \label{eq:upwind conditions}
\end{equation}
and
\begin{equation}
    \left(\frac{\partial \phi}{\partial x}\right)_{i-1} = \frac{\phi_{I} - \phi_{I-1}}{\Delta x}\:.
    \label{eq:cds gradient}
\end{equation}
To set up the temporal discretization, we begin with the transient convection-diffusion equation.
\begin{equation}
    \int\limits_{\Delta t}\frac{\partial}{\partial t}\left( \int\limits_{CV}\rho\phi dV \right)dt + \int\limits_{\Delta t}\:\int\limits_{S}\rho u_x \phi \cdot \overline{n}\:dS\:dt = \int\limits_{\Delta t}\:\int\limits_{S} \Gamma \frac{\partial \phi}{\partial x} \cdot \overline{n}\:dS\:dt
    \label{eq:trans conv diff}
\end{equation}
We can apply the surface integral approximations to get:
\begin{equation*}
    \int\limits_{\Delta t}\frac{\partial}{\partial t}\left( \int\limits_{CV}\rho\phi dV \right)dt + \int\limits_{\Delta t}\left[ \rho u_x S\left( \phi_{i+1} - \phi_{i-1} \right) \right]\:dt = \int\limits_{\Delta t}\Gamma S\left[ \left( \frac{\partial \phi}{\partial x} \right)_{i+1} - \left( \frac{\partial \phi}{\partial x} \right)_{i-1} \right] dt\:.
\end{equation*}
Performing the rest of the integrations, we get:
\begin{equation}
    \rho V \left( \phi_{I}^{n+1} - \phi_{I}^{n} \right) + \rho u_x S \left( \phi_{i+1} - \phi_{i-1} \right)\:\Delta t = \Gamma S \left[ \left( \frac{\partial \phi}{\partial x} \right)_{i+1} - \left( \frac{\partial \phi}{\partial x} \right)_{i-1} \right]\Delta t\:.
    \label{eq:trans disc}
\end{equation}
We can apply \Cref{eq:upwind conditions} and \Cref{eq:cds gradient} for positive velocity. We can also divide by the surface to get:
\begin{equation}
    \rho\: \Delta x \left( \phi_{I}^{n+1} - \phi_{I}^{n} \right) + \rho u_x \Delta t \left( \phi_{I}^{X} - \phi_{I-1}^{X} \right) = \Gamma\:\Delta t \left[ \left( \frac{\phi_{I+1}^{X} - \phi_{I}^{X}}{\Delta x} \right) - \left( \phi_{I}^{X} - \phi_{I-1}^{X} \right) \right]\:.
    \label{eq:trans upwind}
\end{equation}
We have three different time discretizations for this problem: Explicit Euler, Implicit Euler, and Trapezoidal. We can use \Cref{eq:trans upwind} for each scheme's inner nodes, but the boundary nodes will need to use \Cref{eq:trans disc} since they have different gradients.


\subsection{Explicit Euler}
Explicit Euler uses the substitution: $\phi^{X} = \phi^{n}$.
Inner Nodes:
\begin{equation}
    \phi_{I}^{n+1} = \left[ \frac{u_x\: \Delta t}{\Delta x} + \frac{\Gamma\: \Delta t}{\rho\: \left(\Delta x)^2\right)} \right] \phi_{I-1}^{n} + \left[ 1 - \frac{u_x\: \Delta t}{\Delta x} - \frac{2\Gamma\: \Delta t}{\rho\: \left(\Delta x\right)^2} \right] \phi_{I}^{n} + \left[ \frac{\Gamma\:\Delta t}{\rho\: \left(\Delta x\right)^2} \right]\phi_{I+1}^{n}
    \label{eq:ee inner}
\end{equation}
Left Node (Node 1): $\phi_{i-1} = \phi_L$ and $\left( \frac{\partial \phi}{\partial x} \right)_{i-1} = \frac{\phi_I - \phi_L}{\Delta x/2}$
\begin{equation}
    \phi_{I}^{n+1} = \left[ \frac{u_x\: \Delta t}{\Delta x} \right]\phi_{L} + \left[ 1 - \frac{u_x\:\Delta t}{\Delta x} - \frac{3\Gamma\:\Delta t}{\rho\left(\Delta x\right)^2} \right]\phi_{I}^{n} + \left[ \frac{\Gamma\:\Delta t}{\Delta x} \right]\phi_{I+1}^{n}
    \label{eq:ee left}
\end{equation}
Right Node (Node N): $\phi_{i+1} = \phi_R$ and $\left( \frac{\partial \phi}{\partial x} \right)_{i+1} = \frac{\phi_R - \phi_I}{\Delta x/2}$
\begin{equation}
    \phi_{I}^{n+1} = \left[ \frac{u_x\:\Delta t}{\Delta x} + \frac{\Gamma\:\Delta t}{\rho\left(\Delta x\right)^2} \right]\phi_{I-1}^{n} + \left[ 1 - \frac{u_x\:\Delta t}{\Delta x} - \frac{3\Gamma\:\Delta t}{\rho\left(\Delta x\right)^2} \right]\phi_{I}^{n} + \left[ \frac{2\Gamma\:\Delta t}{\rho\left(\Delta x\right)^2} \right]\phi_{R}
    \label{eq:ee right}
\end{equation}


\subsection{Implicit Euler}
Implicit Euler uses the substitution: $\phi^{X} = \phi^{n+1}$.
Inner Nodes:
\begin{equation}
    \left[ \frac{-u_x\: \Delta t}{\Delta x} - \frac{\Gamma\:\Delta t}{\rho\left(\Delta x\right)^2} \right]\phi_{I-1}^{n+1} + \left[ 1 + \frac{u_x\: \Delta t}{\Delta x} + \frac{2\Gamma\: \Delta t}{\rho\left(\Delta x\right)^2} \right]\phi_{I}^{n+1} + \left[ -\frac{\Gamma\: \Delta t}{\rho\left(\Delta x\right)^2} \right]\phi_{I+1}^{n+1} = \phi_{I}^{n}
    \label{eq:ie inner}
\end{equation}
Left Node (Node 1): $\phi_{i-1} = \phi_L$ and $\left( \frac{\partial \phi}{\partial x} \right)_{i-1} = \frac{\phi_I - \phi_L}{\Delta x/2}$
\begin{equation}
    \left[ 1 + \frac{u_x\:\Delta t}{\Delta x} + \frac{3\Gamma\:\Delta t}{\rho\left(\Delta x\right)^2} \right]\phi_{I}^{n+1} + \left[ -\frac{\Gamma\:\Delta t}{\rho\left(\Delta x\right)^2} \right]\phi_{I+1}^{n+1} = \phi_{I}^{n} + \left[ \frac{u_x\:\Delta t}{\Delta x} + \frac{2\Gamma\:\Delta t}{\rho\left(\Delta x\right)^2} \right]\phi_{L}
    \label{eq:ie left}
\end{equation}
Right Node (Node N): $\phi_{i+1} = \phi_R$ and $\left( \frac{\partial \phi}{\partial x} \right)_{i+1} = \frac{\phi_R - \phi_I}{\Delta x/2}$
\begin{equation}
    \left[ -\frac{u_x\:\Delta t}{\Delta x} - \frac{\Gamma\:\Delta t}{\rho\left(\Delta x\right)^2} \right]\phi_{I-1}^{n+1} + \left[ 1 + \frac{u_x\:\Delta t}{\Delta x} + \frac{3\Gamma\:\Delta t}{\rho\left(\Delta x\right)^2} \right]\phi_{I}^{n+1} = \phi_{I}^{n} + \left[ \frac{2\Gamma\:\Delta t}{\rho\left(\Delta x\right)^2} \right]\phi_{R}
    \label{eq:ie right}
\end{equation}


% \subsection{Trapezoidal}
% \begin{multline*}
%     \left[ -\frac{\rho u_x\: \Delta t}{2} - \frac{\Gamma\: \Delta t}{2\: \Delta x} \right]\phi_{I-1}^{n+1} + \left[ \rho \:\Delta x + \frac{\rho u_x\: \Delta t}{2} + \frac{\Gamma\: \Delta t}{\Delta x} \right]\phi_{I}^{n+1} + \left[ -\frac{\Gamma\: \Delta t}{2\: \Delta x} \right]\phi_{I+1}^{n+1}\\
%     = \left[ \frac{\rho u_x\: \Delta t}{2} + \frac{\Gamma\: \Delta t}{2\: \Delta x} \right]\phi_{I-1}^{n} + \left[ \rho \:\Delta x - \frac{\rho u_x\: \Delta t}{2} - \frac{\Gamma\: \Delta t}{\Delta x} \right]\phi_{I}^{n} + \left[ \frac{\Gamma\: \Delta t}{2\: \Delta x} \right]\phi_{I+1}^{n}
% \end{multline*}



\clearpage
\section{Results}

\subsection{Explict Euler}
\Cref{fig:ee1,,fig:ee2,,fig:ee3} show the results for the Explicit Euler solution using $K=0.2,2.0$, and 20.0, respectively. \Cref{tab:error ee} shows the norm error for all Explicit Euler cases.
\begin{figure}[htbp]
    \centering
    \includegraphics[width=\textwidth]{plots/graph_EE_case1.pdf}
    \caption{Explicit Euler solution for case 1: $K=0.2$.}
    \label{fig:ee1}
\end{figure}

\begin{figure}[htbp]
    \centering
    \includegraphics[width=\textwidth]{plots/graph_EE_case2.pdf}
    \caption{Explicit Euler solution for case 2: $K=2.0$.}
    \label{fig:ee2}
\end{figure}

\begin{figure}[htbp]
    \centering
    \includegraphics[width=\textwidth]{plots/graph_EE_case3.pdf}
    \caption{Explicit Euler solution for case 3: $K=20.0$.}
    \label{fig:ee3}
\end{figure}

Norm:
\begin{table}[htbp]
	 \centering
	 \caption{Norm values for Explicit Euler.}
	 \begin{tabular}{cc}
		 \toprule
		 $K$ & Norm \\ 
		 \midrule 
		 0.2 & 1.55418029575927 \\ 
		 2.0 & 8.3196861106867e+245 \\ 
		 20.0 & inf \\ 
		 \bottomrule 
	 \end{tabular} 
	 \label{tab:error ee} 
\end{table}


\clearpage
\subsection{Implicit Euler}
\Cref{fig:ie1,,fig:ie2,,fig:ie3} show the results for the Explicit Euler solution using $K=0.2,2.0$, and 20.0, respectively. \Cref{tab:error ie} shows the norm error for all Explicit Euler cases.
\begin{figure}[htbp]
    \centering
    \includegraphics[width=\textwidth]{plots/graph_IE_case1.pdf}
    \caption{Implicit Euler solution for case 1: $K=0.2$.}
    \label{fig:ie1}
\end{figure}

\begin{figure}[htbp]
    \centering
    \includegraphics[width=\textwidth]{plots/graph_IE_case2.pdf}
    \caption{Implicit Euler solution for case 2: $K=2.0$.}
    \label{fig:ie2}
\end{figure}

\begin{figure}[htbp]
    \centering
    \includegraphics[width=\textwidth]{plots/graph_IE_case3.pdf}
    \caption{Implicit Euler solution for case 3: $K=20.0$.}
    \label{fig:ie3}
\end{figure}


Norm:
\begin{table}[htbp]
	 \centering
	 \caption{Norm values for Implicit Euler.}
	 \begin{tabular}{cc}
		 \toprule
		 $K$ & Norm \\ 
		 \midrule 
		 0.2 & 1.5567368462357045 \\ 
		 2.0 & 1.5504768792236276 \\ 
		 20.0 & 1.5504768792236157 \\ 
		 \bottomrule 
	 \end{tabular} 
	 \label{tab:error ie} 
\end{table}


\clearpage
\section{Discussion}
For the Explicit Euler results, we see that for other than the smallest time step, the solution is unbounded. We can perform a stability analysis to see what the time step needs to be for Explicit Euler to be stable. We will begin by writing the inner node equation in terms of the time step to characteristic diffusion time ratio, $d=\frac{\Gamma\:\Delta t}{\rho\left(\Delta x\right)^2}$; and the Courant number, $c=\frac{u_x\:\Delta t}{\Delta x}$.
\begin{equation*}
    \phi_{I}^{n+1} = \left[ c + d \right]\phi_{I-1}^{n} + \left[ 1-c-2d \right]\phi_{I}^{n} + \left[ d \right]\phi_{I+1}^{n}
\end{equation*}
Then we can substitute the eigenvalue into the equation.
\begin{equation*}
    \sigma^{n+1} e^{i\alpha I} = \left[ c + d \right]\sigma^{n} e^{i\alpha \left(I-1\right)} + \left[ 1-c-2d \right]\sigma^{n} e^{i\alpha I} + \left[ d \right]\sigma^{n} e^{i\alpha \left(I+1\right)}
\end{equation*}
We can divide by $\sigma^n$ and $e^{i\alpha I}$.
\begin{equation*}
    \sigma = \left[ c + d \right] e^{-i\alpha} + \left[ 1-c-2d \right] + \left[ d \right] e^{i\alpha}
\end{equation*}
We can then substitute the identity for $e^{i\alpha}$ and rearrange to get the solution for the eigenvalue.
\begin{equation}
    \sigma = 1 + c \left( \cos\left( \alpha \right) - 1 \right) + 2d \left( \cos\left( \alpha \right) - 1 \right) - ic\sin\left( \alpha \right)
    \label{eq:eigenvalue}
\end{equation}
To be stable, $\sigma < 1$ and $\sigma^2 < 1$ for all values of $\alpha$. We can find the bounding values for $c$ and $d$ by looking at the extreme cases. Firstly, when $d=0$:
\begin{equation*}
    \sigma = 1 + c \left( \cos\left( \alpha \right) - 1 \right) - ic\sin\left( \alpha \right)
\end{equation*}
\begin{equation*}
    \sigma^2 = \left[ 1 + c \left( \cos\left( \alpha \right) - 1 \right) \right]^2 + c^2\sin^2\left( \alpha \right)
\end{equation*}
We can see that for any value of $c$, $sigma^2 < 1$, so this is always unstable. We do not get any restrictions on $c$ from this case. Next is when $c=0$.
\begin{equation*}
    \sigma = 1 + 2d \left( \cos\left( \alpha \right) - 1 \right)
\end{equation*}
The worst case is when $\cos\left( \alpha \right) = -1$.
\begin{equation*}
    1 > 1 + 2d \left( -1 - 1 \right)
\end{equation*}
\begin{equation*}
    d < \frac{1}{2}
\end{equation*}
Now for $\sigma^2$ with the same conditions:
\begin{equation*}
    \sigma^2 = \left[1 + 2d \left( \cos\left( \alpha \right) - 1 \right)\right]^2
\end{equation*}
\begin{equation*}
    1 > \left[1 + 2d \left( -1 - 1 \right)\right]^2
\end{equation*}
\begin{equation*}
    1 > \left[1 - 4d\right]^2
\end{equation*}
\begin{equation*}
    1 > 1 - 8d + 16d^2
\end{equation*}
\begin{equation*}
    d < \frac{1}{2}
\end{equation*}
So both conditions come to the same bounds that $d$ must be less than 1/2. \Cref{tab:diff ratio} shows $d$ for each time step size and we can see that only the smallest time step is less than 1/2 and the others will not be stable. This matches up with the results as the plots for the second two solutions for Explicit Euler show the instability.

\begin{table}[htbp]
	 \centering
	 \caption{Time step to characteristic diffusion time ratio.}
	 \begin{tabular}{cc}
		 \toprule
		 $K$ & $d$ \\ 
		 \midrule 
		 0.2 & 0.16000000000000003 \\ 
		 2.0 & 1.5999999999999996 \\ 
		 20.0 & 16.0 \\ 
		 \bottomrule 
	 \end{tabular} 
	 \label{tab:diff ratio} 
\end{table}

The results for Implicit Euler are all stable since Implicit Euler is an unconditionally stable solution method. We can also see that for the smallest time step, we can see the convergence over time towards the steady solution more clearly. The largest time step gets to the steady soltuion almost immediately. This is expected since we are plotting the solution at certain time steps, not certain times. So the solutions with larger time steps are closer to the steady solution in time at the same time step number than with smaller time steps.


\clearpage
\section{Code}
\pythonexternal{code/main.py}


\end{document}
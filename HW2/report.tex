\documentclass[12pt]{article}
\usepackage{fancyhdr}
\usepackage[letterpaper, margin=1in]{geometry}
%\usepackage{indentfirst}
\usepackage{graphicx}
\usepackage{amsmath}
\usepackage{amssymb}
\usepackage{siunitx}
\sisetup{detect-weight=true, detect-family=true} % makes siunitx follow font formatting like bold, italic, etc.
\usepackage{cancel}
\usepackage{isotope}
\usepackage{listings}
\usepackage[dvipsnames,table]{xcolor}
\usepackage{xspace}
\usepackage{booktabs} % makes tables pretty
\usepackage{longtable} % for long tables
\usepackage{multirow} % makes tables pretty
\usepackage{multicol} % makes tables pretty
\usepackage{setspace}
\usepackage{subcaption}
\usepackage{hyperref}
\usepackage{cleveref}
\newcommand{\creflastconjunction}{, and\nobreakspace} % adds oxford comma to cleveref
\usepackage[utf8]{inputenc}
\usepackage{textcomp}
\usepackage{titlesec}
\usepackage{svg}
\usepackage{pdflscape} % makes pages landscape
\usepackage{mathtools}
\usepackage{enumitem}
\usepackage[T1]{fontenc}


% ----------------- Commands ---------------------%
\newcommand{\mathcolorbox}[2]{\colorbox{#1}{$\displaystyle #2$}}
\definecolor{codegreen}{rgb}{0,0.6,0}
\definecolor{codegray}{rgb}{0.5,0.5,0.5}
\definecolor{codepurple}{rgb}{0.58,0,0.82}
\definecolor{backcolour}{rgb}{0.95,0.95,0.92}

% listings stuff for python
% Default fixed font does not support bold face
\DeclareFixedFont{\ttb}{T1}{txtt}{bx}{n}{12} % for bold
\DeclareFixedFont{\ttm}{T1}{txtt}{m}{n}{12}  % for normal

% Custom colors
%\usepackage{color}
\definecolor{deepblue}{rgb}{0,0,0.5}
\definecolor{deepred}{rgb}{0.6,0,0}
\definecolor{deepgreen}{rgb}{0,0.5,0}

%\usepackage{listings}

% Python style for highlighting
\newcommand\pythonstyle{\lstset{
language=Python,
basicstyle=\ttm,
morekeywords={self},              % Add keywords here
keywordstyle=\ttb\color{deepblue},
emph={MyClass,__init__},          % Custom highlighting
emphstyle=\ttb\color{deepred},    % Custom highlighting style
stringstyle=\color{deepgreen},
commentstyle=\color{codegray},
backgroundcolor = \color{backcolour},
breaklines=true,
numbers=left,
numberstyle=\small\color{codegray},
frame=tb,                         % Any extra options here
showstringspaces=false
}}

% Python for external files
\newcommand\pythonexternal[2][]{{
\pythonstyle
\lstinputlisting[#1]{#2}}}




% si units stuff
\DeclareSIUnit\year{yr}
\DeclareSIUnit\hour{hr}
\DeclareSIUnit\mole{mol}

% fancy header stuff
\usepackage{fancyhdr}
\pagestyle{fancy}

\setlength{\headheight}{28pt}
\lhead{NSE 565 \\ Winter 2022}
\chead{Homework 2\\}
\rhead{Austin Warren\\Due March 11, 2022}

% bib if needed
\bibliographystyle{ieeetr}

\begin{document}

%%%%%%%%%%%%%%%%%%%%%%%%%%%%%%%%%%
\section{Methods}
First, we will prepare the Upwind scheme for the spatial discretization. The conditions for the Upwind scheme are listed below:
\begin{equation*}
    \phi_{i+1} = \begin{cases}
                    \phi_{I} & \overline{V} > 0\\
                    \phi_{I+1} & \overline{V} < 0
                 \end{cases}\:,
\end{equation*}
and
\begin{equation}
    \left(\frac{\partial \phi}{\partial x}\right)_{i-1} = \frac{\phi_{I} - \phi_{I-1}}{\Delta x}\:.
\end{equation}
Discretizing gives:
\begin{equation*}
    \left( \rho u_x \right)_{i+1} \phi_{i+1} - \left( \rho u_x \right)_{i-1} \phi_{i-1} = \Gamma \left( \frac{\partial \phi}{\partial x} \right)_{i+1} - \Gamma \left( \frac{\partial \phi}{\partial x} \right)_{i-1}
\end{equation*}
Plugging in our Upwind scheme conditions for velocity greater than zero gives:
\begin{equation*}
    \left[-2F - D  \right] \phi_{I-1} + \left[ 2F + 2D \right] \phi_{I} + \left[ D \right] \phi_{I+1} = 0\:,
\end{equation*}
where
\begin{equation*}
    D = \frac{\Gamma}{\Delta x} ~~~~~ F = \frac{\rho u_x}{2}\:.
\end{equation*}
This equation works for the interior nodes. For the boundary nodes, we will need different equations. For the left boundary:
\begin{equation*}
    \left[ 2F + 2D \right] \phi_{I} + \left[ D \right] \phi_{I+1} = \left[ 2F + D \right] \phi_{L}\:.
\end{equation*}
For the right boundary:
\begin{equation*}
    \left[ -D \right] \phi_{I-1} + \left[ -2F - 2D \right] \phi_{I} = \left[ -2F + D \right] \phi_{R}\:.
\end{equation*}
We have three different time discretizations for this problem: Explicit Euler, Implicit Euler, and Trapezoidal.
Explicit Euler:
% \begin{equation*}
%     \phi_{I}^{n+1} = \frac{1}{\rho\: \Delta x}\left[ \rho u_x\: \Delta t + \frac{\Gamma\: \Delta t}{\Delta x}\right] \phi_{I-1}^{n} + \frac{1}{\rho\:\Delta x}\left[ \rho \:\Delta x - \rho u_x\: \Delta t - \frac{2\Gamma\: \Delta t}{\Delta x} \right] \phi_{I}^{n} + \frac{1}{\rho\:\Delta x}\left[ \frac{\Gamma\:\Delta t}{\Delta x} \right]\phi_{I+1}^{n}
% \end{equation*}
\begin{equation*}
    \phi_{I}^{n+1} = \left[ \frac{u_x\: \Delta t}{\Delta x} + \frac{\Gamma\: \Delta t}{\rho\: \left(\Delta x)^2\right)} \right] \phi_{I-1}^{n} + \left[ 1 - \frac{u_x\: \Delta t}{\Delta x} - \frac{2\Gamma\: \Delta t}{\rho\: \left(\Delta x\right)^2} \right] \phi_{I}^{n} + \left[ \frac{\Gamma\:\Delta t}{\rho\: \left(\Delta x\right)^2} \right]\phi_{I+1}^{n}
\end{equation*}
Implicit Euler:
\begin{equation*}
    \frac{1}{\rho \:\Delta x}\left[ -\rho u_x\: \Delta t \right]\phi_{I-1}^{n+1} + \frac{1}{\rho \:\Delta x}\left[ \rho \:\Delta x + \rho u_x\: \Delta t + 2\frac{\Gamma\: \Delta t}{\Delta x} \right]\phi_{I}^{n+1} + \frac{1}{\rho \:\Delta x}\left[ -\frac{\Gamma\: \Delta t}{\Delta x} \right]\phi_{I+1}^{n+1} = \phi_{I}^{n}
\end{equation*}
Trapezoidal:
\begin{multline*}
    \left[ -\frac{\rho u_x\: \Delta t}{2} - \frac{\Gamma\: \Delta t}{2\: \Delta x} \right]\phi_{I-1}^{n+1} + \left[ \rho \:\Delta x + \frac{\rho u_x\: \Delta t}{2} + \frac{\Gamma\: \Delta t}{\Delta x} \right]\phi_{I}^{n+1} + \left[ -\frac{\Gamma\: \Delta t}{2\: \Delta x} \right]\phi_{I+1}^{n+1}\\
    = \left[ \frac{\rho u_x\: \Delta t}{2} + \frac{\Gamma\: \Delta t}{2\: \Delta x} \right]\phi_{I-1}^{n} + \left[ \rho \:\Delta x - \frac{\rho u_x\: \Delta t}{2} - \frac{\Gamma\: \Delta t}{\Delta x} \right]\phi_{I}^{n} + \left[ \frac{\Gamma\: \Delta t}{2\: \Delta x} \right]\phi_{I+1}^{n}
\end{multline*}









\end{document}